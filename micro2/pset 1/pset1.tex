% Options for packages loaded elsewhere
\PassOptionsToPackage{unicode}{hyperref}
\PassOptionsToPackage{hyphens}{url}
\documentclass[
]{article}
\usepackage{xcolor}
\usepackage[margin=1in]{geometry}
\usepackage{amsmath,amssymb}
\setcounter{secnumdepth}{-\maxdimen} % remove section numbering
\usepackage{iftex}
\ifPDFTeX
  \usepackage[T1]{fontenc}
  \usepackage[utf8]{inputenc}
  \usepackage{textcomp} % provide euro and other symbols
\else % if luatex or xetex
  \usepackage{unicode-math} % this also loads fontspec
  \defaultfontfeatures{Scale=MatchLowercase}
  \defaultfontfeatures[\rmfamily]{Ligatures=TeX,Scale=1}
\fi
\usepackage{lmodern}
\ifPDFTeX\else
  % xetex/luatex font selection
\fi
% Use upquote if available, for straight quotes in verbatim environments
\IfFileExists{upquote.sty}{\usepackage{upquote}}{}
\IfFileExists{microtype.sty}{% use microtype if available
  \usepackage[]{microtype}
  \UseMicrotypeSet[protrusion]{basicmath} % disable protrusion for tt fonts
}{}
\makeatletter
\@ifundefined{KOMAClassName}{% if non-KOMA class
  \IfFileExists{parskip.sty}{%
    \usepackage{parskip}
  }{% else
    \setlength{\parindent}{0pt}
    \setlength{\parskip}{6pt plus 2pt minus 1pt}}
}{% if KOMA class
  \KOMAoptions{parskip=half}}
\makeatother
\usepackage{graphicx}
\makeatletter
\newsavebox\pandoc@box
\newcommand*\pandocbounded[1]{% scales image to fit in text height/width
  \sbox\pandoc@box{#1}%
  \Gscale@div\@tempa{\textheight}{\dimexpr\ht\pandoc@box+\dp\pandoc@box\relax}%
  \Gscale@div\@tempb{\linewidth}{\wd\pandoc@box}%
  \ifdim\@tempb\p@<\@tempa\p@\let\@tempa\@tempb\fi% select the smaller of both
  \ifdim\@tempa\p@<\p@\scalebox{\@tempa}{\usebox\pandoc@box}%
  \else\usebox{\pandoc@box}%
  \fi%
}
% Set default figure placement to htbp
\def\fps@figure{htbp}
\makeatother
\setlength{\emergencystretch}{3em} % prevent overfull lines
\providecommand{\tightlist}{%
  \setlength{\itemsep}{0pt}\setlength{\parskip}{0pt}}
\usepackage{bookmark}
\IfFileExists{xurl.sty}{\usepackage{xurl}}{} % add URL line breaks if available
\urlstyle{same}
\hypersetup{
  pdftitle={Problem set 1},
  pdfauthor={Jay Ballesteros},
  hidelinks,
  pdfcreator={LaTeX via pandoc}}

\title{Problem set 1}
\usepackage{etoolbox}
\makeatletter
\providecommand{\subtitle}[1]{% add subtitle to \maketitle
  \apptocmd{\@title}{\par {\large #1 \par}}{}{}
}
\makeatother
\subtitle{PPHA 32400 Microeconomics and Public Policy II}
\author{Jay Ballesteros}
\date{January 13, 2026}

\begin{document}
\maketitle

\subsection{1. Consider a profit-maximizing firm with a marginal cost
function given by MC(q) = 10 + q. It is a perfectly competitive firm
i.e.~it is a price-taker. The market price is
p.}\label{consider-a-profit-maximizing-firm-with-a-marginal-cost-function-given-by-mcq-10-q.-it-is-a-perfectly-competitive-firm-i.e.-it-is-a-price-taker.-the-market-price-is-p.}

\textbf{a) Consider a situation where the firm is already in the market.
What is the optimal quantity to produce? (Hint: the optimal quantity
will be a function of p.)}

\textbf{b) Consider a situation where the firm is already in the market.
Depict the marginal cost curve graphically, with the quantity produced
on the x-axis and the prices/costs on the y-axis. Label the intercepts,
as well as the optimal quantity. Label the producer surplus. Assume that
the firm is already in the market. What is the producer surplus earned
by the firm? You may assume that p \textgreater{} 10. (Hint: The
producer surplus will be a function of p.)}

\textbf{c) Consider a situation where the the firm is deciding whether
to enter the market or not. Suppose that the fixed cost is FC= 50. What
is the price below which the firm will not enter the market. (Hint: the
firm will enter the market if it's profit π is positive. Profit is
producer surplus minus the fixed cost i.e.~π = PS−FC}

\textbf{d) If the fixed cost is FC= 50 (as it was in the previous
question) what is the firm's supply function? Depict it graphically,
with the price on the y-axis, and quantity on the x-axis. Label any
kinks in the curve}

\end{document}
